% Options for packages loaded elsewhere
\PassOptionsToPackage{unicode}{hyperref}
\PassOptionsToPackage{hyphens}{url}
%
\documentclass[
]{article}
\usepackage{amsmath,amssymb}
\usepackage{lmodern}
\usepackage{ifxetex,ifluatex}
\ifnum 0\ifxetex 1\fi\ifluatex 1\fi=0 % if pdftex
  \usepackage[T1]{fontenc}
  \usepackage[utf8]{inputenc}
  \usepackage{textcomp} % provide euro and other symbols
\else % if luatex or xetex
  \usepackage{unicode-math}
  \defaultfontfeatures{Scale=MatchLowercase}
  \defaultfontfeatures[\rmfamily]{Ligatures=TeX,Scale=1}
\fi
% Use upquote if available, for straight quotes in verbatim environments
\IfFileExists{upquote.sty}{\usepackage{upquote}}{}
\IfFileExists{microtype.sty}{% use microtype if available
  \usepackage[]{microtype}
  \UseMicrotypeSet[protrusion]{basicmath} % disable protrusion for tt fonts
}{}
\makeatletter
\@ifundefined{KOMAClassName}{% if non-KOMA class
  \IfFileExists{parskip.sty}{%
    \usepackage{parskip}
  }{% else
    \setlength{\parindent}{0pt}
    \setlength{\parskip}{6pt plus 2pt minus 1pt}}
}{% if KOMA class
  \KOMAoptions{parskip=half}}
\makeatother
\usepackage{xcolor}
\IfFileExists{xurl.sty}{\usepackage{xurl}}{} % add URL line breaks if available
\IfFileExists{bookmark.sty}{\usepackage{bookmark}}{\usepackage{hyperref}}
\hypersetup{
  hidelinks,
  pdfcreator={LaTeX via pandoc}}
\urlstyle{same} % disable monospaced font for URLs
\usepackage[margin=1in]{geometry}
\usepackage{color}
\usepackage{fancyvrb}
\newcommand{\VerbBar}{|}
\newcommand{\VERB}{\Verb[commandchars=\\\{\}]}
\DefineVerbatimEnvironment{Highlighting}{Verbatim}{commandchars=\\\{\}}
% Add ',fontsize=\small' for more characters per line
\usepackage{framed}
\definecolor{shadecolor}{RGB}{248,248,248}
\newenvironment{Shaded}{\begin{snugshade}}{\end{snugshade}}
\newcommand{\AlertTok}[1]{\textcolor[rgb]{0.94,0.16,0.16}{#1}}
\newcommand{\AnnotationTok}[1]{\textcolor[rgb]{0.56,0.35,0.01}{\textbf{\textit{#1}}}}
\newcommand{\AttributeTok}[1]{\textcolor[rgb]{0.77,0.63,0.00}{#1}}
\newcommand{\BaseNTok}[1]{\textcolor[rgb]{0.00,0.00,0.81}{#1}}
\newcommand{\BuiltInTok}[1]{#1}
\newcommand{\CharTok}[1]{\textcolor[rgb]{0.31,0.60,0.02}{#1}}
\newcommand{\CommentTok}[1]{\textcolor[rgb]{0.56,0.35,0.01}{\textit{#1}}}
\newcommand{\CommentVarTok}[1]{\textcolor[rgb]{0.56,0.35,0.01}{\textbf{\textit{#1}}}}
\newcommand{\ConstantTok}[1]{\textcolor[rgb]{0.00,0.00,0.00}{#1}}
\newcommand{\ControlFlowTok}[1]{\textcolor[rgb]{0.13,0.29,0.53}{\textbf{#1}}}
\newcommand{\DataTypeTok}[1]{\textcolor[rgb]{0.13,0.29,0.53}{#1}}
\newcommand{\DecValTok}[1]{\textcolor[rgb]{0.00,0.00,0.81}{#1}}
\newcommand{\DocumentationTok}[1]{\textcolor[rgb]{0.56,0.35,0.01}{\textbf{\textit{#1}}}}
\newcommand{\ErrorTok}[1]{\textcolor[rgb]{0.64,0.00,0.00}{\textbf{#1}}}
\newcommand{\ExtensionTok}[1]{#1}
\newcommand{\FloatTok}[1]{\textcolor[rgb]{0.00,0.00,0.81}{#1}}
\newcommand{\FunctionTok}[1]{\textcolor[rgb]{0.00,0.00,0.00}{#1}}
\newcommand{\ImportTok}[1]{#1}
\newcommand{\InformationTok}[1]{\textcolor[rgb]{0.56,0.35,0.01}{\textbf{\textit{#1}}}}
\newcommand{\KeywordTok}[1]{\textcolor[rgb]{0.13,0.29,0.53}{\textbf{#1}}}
\newcommand{\NormalTok}[1]{#1}
\newcommand{\OperatorTok}[1]{\textcolor[rgb]{0.81,0.36,0.00}{\textbf{#1}}}
\newcommand{\OtherTok}[1]{\textcolor[rgb]{0.56,0.35,0.01}{#1}}
\newcommand{\PreprocessorTok}[1]{\textcolor[rgb]{0.56,0.35,0.01}{\textit{#1}}}
\newcommand{\RegionMarkerTok}[1]{#1}
\newcommand{\SpecialCharTok}[1]{\textcolor[rgb]{0.00,0.00,0.00}{#1}}
\newcommand{\SpecialStringTok}[1]{\textcolor[rgb]{0.31,0.60,0.02}{#1}}
\newcommand{\StringTok}[1]{\textcolor[rgb]{0.31,0.60,0.02}{#1}}
\newcommand{\VariableTok}[1]{\textcolor[rgb]{0.00,0.00,0.00}{#1}}
\newcommand{\VerbatimStringTok}[1]{\textcolor[rgb]{0.31,0.60,0.02}{#1}}
\newcommand{\WarningTok}[1]{\textcolor[rgb]{0.56,0.35,0.01}{\textbf{\textit{#1}}}}
\usepackage{graphicx}
\makeatletter
\def\maxwidth{\ifdim\Gin@nat@width>\linewidth\linewidth\else\Gin@nat@width\fi}
\def\maxheight{\ifdim\Gin@nat@height>\textheight\textheight\else\Gin@nat@height\fi}
\makeatother
% Scale images if necessary, so that they will not overflow the page
% margins by default, and it is still possible to overwrite the defaults
% using explicit options in \includegraphics[width, height, ...]{}
\setkeys{Gin}{width=\maxwidth,height=\maxheight,keepaspectratio}
% Set default figure placement to htbp
\makeatletter
\def\fps@figure{htbp}
\makeatother
\setlength{\emergencystretch}{3em} % prevent overfull lines
\providecommand{\tightlist}{%
  \setlength{\itemsep}{0pt}\setlength{\parskip}{0pt}}
\setcounter{secnumdepth}{-\maxdimen} % remove section numbering
\usepackage{booktabs}
\usepackage{longtable}
\usepackage{array}
\usepackage{multirow}
\usepackage{wrapfig}
\usepackage{float}
\usepackage{colortbl}
\usepackage{pdflscape}
\usepackage{tabu}
\usepackage{threeparttable}
\usepackage{threeparttablex}
\usepackage[normalem]{ulem}
\usepackage{makecell}
\usepackage{xcolor}
\ifluatex
  \usepackage{selnolig}  % disable illegal ligatures
\fi

\author{}
\date{\vspace{-2.5em}}

\begin{document}

\ifdefined\ifprincipal
\else
\setlength{\parindent}{1em}
\pagestyle{fancy}
\setcounter{tocdepth}{4}
\tableofcontents

\fi

\ifdefined\ifdoblecara
\fancyhead{}{}
\fancyhead[LE,RO]{\scriptsize\rightmark}
\fancyfoot[LO,RE]{\scriptsize\slshape \leftmark}
\fancyfoot[C]{}
\fancyfoot[LE,RO]{\footnotesize\thepage}
\else
\fancyhead{}{}
\fancyhead[RO]{\scriptsize\rightmark}
\fancyfoot[LO]{\scriptsize\slshape \leftmark}
\fancyfoot[C]{}
\fancyfoot[RO]{\footnotesize\thepage}
\fi
\renewcommand{\headrulewidth}{0.4pt}
\renewcommand{\footrulewidth}{0.4pt}

\hypertarget{proceso-de-la-ciencia-de-datos}{%
\subsection{Proceso de la ciencia de
datos}\label{proceso-de-la-ciencia-de-datos}}

Este apartado lo dedicaremos a realizar un proceso de ciencia de datos
completo, teniendo en cuenta los siguientes objetivos:

\begin{itemize}
\tightlist
\item
  Analizar los datos proporcionados para conocer como varían las ventas
  de productos lácteos con el tiempo
\item
  Demostrar que existe la posibilidad de construir buenos modelos para
  predecir el volumen futuro de venta de productos a partir de los datos
\item
  Desarrollo de modelos para predecir ventas
\end{itemize}

\hypertarget{lectura-y-descripciuxf3n-de-los-datos}{%
\subsubsection{Lectura y descripción de los
datos}\label{lectura-y-descripciuxf3n-de-los-datos}}

Los datos contienen información correspondiente a ventas de dos
productos lácteos (uno con y sin calcio) durante un período de 5 meses,
desde el 1 de Septiembre de 2020 hasta el 30 de Enero de 2021,
obteniéndose un total de 140025 observaciones y se estructuran de la
siguiente forma: Cada fila corresponde a la línea de un ticket y hace
referencia a la venta de un artículo en particular.

En este conjunto de datos inicial encontramos las siguientes variables:

\begin{itemize}
\tightlist
\item
  \textbf{Id de ticket}: Variable numérica que identifica unívocamente a
  cada ticket de venta.
\item
  \textbf{Línea de ticket}: Variable numérica con la línea
  correspondiente del ticket.
\item
  \textbf{Fecha}: Fecha en que se realizó la venta.
\item
  \textbf{Código}: Identificador del producto.
\item
  \textbf{Cantidad}: Número de items vendidos de un determinado
  producto.
\item
  \textbf{Precio}: Precio base del artículo libre de impuestos, euros.
\item
  \textbf{Precio con impuestos}: Precio de venta del artículo, en euros.
\item
  \textbf{Descuento}: Descuento aplicado.
\item
  \textbf{Importe}: Importe de la compra libre de impuestos, en euros.
\item
  \textbf{Importe con impuestos}: Importe a pagar por el comprador, en
  euros.
\end{itemize}

\hypertarget{preparaciuxf3n-de-los-datos-preprocesado}{%
\subsubsection{Preparación de los datos
(Preprocesado)}\label{preparaciuxf3n-de-los-datos-preprocesado}}

En este paso, vamos a llevar a cabo la limpieza de los datos para su
posterior estudio, representación y modelado. En este punto del proceso,
trataremos de encontrar, corregir o eliminar registros erróneos en los
datos.

\hypertarget{transformaciuxf3n-de-los-datos}{%
\paragraph{Transformación de los
datos}\label{transformaciuxf3n-de-los-datos}}

Hay algunas variables que necesitan ser transformadas, en particular, la
variable código, la cantidad de artículos vendidos y linea del ticket
han sido transformadas para tenerlas en un formato adecuado. Vamos a
visualizar la nueva estructura de los datos:

\footnotesize

\begin{Shaded}
\begin{Highlighting}[]
\NormalTok{dataset }\SpecialCharTok{\%\textgreater{}\%} \FunctionTok{str}\NormalTok{() }\CommentTok{\# Estructura de los datos trás reformateo}
\end{Highlighting}
\end{Shaded}

\begin{verbatim}
## 'data.frame':    140025 obs. of  10 variables:
##  $ ID_TICKET            : num  22549194 22549215 22549242 22549242 22549264 ...
##  $ LINEA_TICKET         : Factor w/ 89 levels "1","2","3","4",..: 1 1 1 2 3 7 1 2 3 6 ...
##  $ FECHA                : Date, format: "2020-08-01" "2020-08-01" ...
##  $ CODIGO               : Factor w/ 2 levels "20445","22336": 2 1 2 2 2 2 2 2 2 1 ...
##  $ CANTIDAD             : int  1 6 6 6 1 1 6 1 1 5 ...
##  $ PRECIO               : num  1.35 1.26 1.35 1.35 1.35 1.35 1.35 1.35 1.35 1.26 ...
##  $ PRECIO_CON_IMPUESTOS : num  1.49 1.39 1.49 1.49 1.49 1.49 1.49 1.49 1.49 1.39 ...
##  $ DESCUENTO            : num  0 0 0 0 0 0 0 0 0 0 ...
##  $ IMPORTE              : num  1.35 7.58 8.13 8.13 1.35 1.35 8.13 1.35 1.35 6.32 ...
##  $ IMPORTE_CON_IMPUESTOS: num  1.49 8.34 8.94 8.94 1.49 1.49 8.94 1.49 1.49 6.95 ...
\end{verbatim}

\normalsize

\end{document}
