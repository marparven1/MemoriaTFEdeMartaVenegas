% Options for packages loaded elsewhere
\PassOptionsToPackage{unicode}{hyperref}
\PassOptionsToPackage{hyphens}{url}
%
\documentclass[
]{article}
\usepackage{amsmath,amssymb}
\usepackage{lmodern}
\usepackage{ifxetex,ifluatex}
\ifnum 0\ifxetex 1\fi\ifluatex 1\fi=0 % if pdftex
  \usepackage[T1]{fontenc}
  \usepackage[utf8]{inputenc}
  \usepackage{textcomp} % provide euro and other symbols
\else % if luatex or xetex
  \usepackage{unicode-math}
  \defaultfontfeatures{Scale=MatchLowercase}
  \defaultfontfeatures[\rmfamily]{Ligatures=TeX,Scale=1}
\fi
% Use upquote if available, for straight quotes in verbatim environments
\IfFileExists{upquote.sty}{\usepackage{upquote}}{}
\IfFileExists{microtype.sty}{% use microtype if available
  \usepackage[]{microtype}
  \UseMicrotypeSet[protrusion]{basicmath} % disable protrusion for tt fonts
}{}
\makeatletter
\@ifundefined{KOMAClassName}{% if non-KOMA class
  \IfFileExists{parskip.sty}{%
    \usepackage{parskip}
  }{% else
    \setlength{\parindent}{0pt}
    \setlength{\parskip}{6pt plus 2pt minus 1pt}}
}{% if KOMA class
  \KOMAoptions{parskip=half}}
\makeatother
\usepackage{xcolor}
\IfFileExists{xurl.sty}{\usepackage{xurl}}{} % add URL line breaks if available
\IfFileExists{bookmark.sty}{\usepackage{bookmark}}{\usepackage{hyperref}}
\hypersetup{
  hidelinks,
  pdfcreator={LaTeX via pandoc}}
\urlstyle{same} % disable monospaced font for URLs
\usepackage[margin=1in]{geometry}
\usepackage{color}
\usepackage{fancyvrb}
\newcommand{\VerbBar}{|}
\newcommand{\VERB}{\Verb[commandchars=\\\{\}]}
\DefineVerbatimEnvironment{Highlighting}{Verbatim}{commandchars=\\\{\}}
% Add ',fontsize=\small' for more characters per line
\usepackage{framed}
\definecolor{shadecolor}{RGB}{248,248,248}
\newenvironment{Shaded}{\begin{snugshade}}{\end{snugshade}}
\newcommand{\AlertTok}[1]{\textcolor[rgb]{0.94,0.16,0.16}{#1}}
\newcommand{\AnnotationTok}[1]{\textcolor[rgb]{0.56,0.35,0.01}{\textbf{\textit{#1}}}}
\newcommand{\AttributeTok}[1]{\textcolor[rgb]{0.77,0.63,0.00}{#1}}
\newcommand{\BaseNTok}[1]{\textcolor[rgb]{0.00,0.00,0.81}{#1}}
\newcommand{\BuiltInTok}[1]{#1}
\newcommand{\CharTok}[1]{\textcolor[rgb]{0.31,0.60,0.02}{#1}}
\newcommand{\CommentTok}[1]{\textcolor[rgb]{0.56,0.35,0.01}{\textit{#1}}}
\newcommand{\CommentVarTok}[1]{\textcolor[rgb]{0.56,0.35,0.01}{\textbf{\textit{#1}}}}
\newcommand{\ConstantTok}[1]{\textcolor[rgb]{0.00,0.00,0.00}{#1}}
\newcommand{\ControlFlowTok}[1]{\textcolor[rgb]{0.13,0.29,0.53}{\textbf{#1}}}
\newcommand{\DataTypeTok}[1]{\textcolor[rgb]{0.13,0.29,0.53}{#1}}
\newcommand{\DecValTok}[1]{\textcolor[rgb]{0.00,0.00,0.81}{#1}}
\newcommand{\DocumentationTok}[1]{\textcolor[rgb]{0.56,0.35,0.01}{\textbf{\textit{#1}}}}
\newcommand{\ErrorTok}[1]{\textcolor[rgb]{0.64,0.00,0.00}{\textbf{#1}}}
\newcommand{\ExtensionTok}[1]{#1}
\newcommand{\FloatTok}[1]{\textcolor[rgb]{0.00,0.00,0.81}{#1}}
\newcommand{\FunctionTok}[1]{\textcolor[rgb]{0.00,0.00,0.00}{#1}}
\newcommand{\ImportTok}[1]{#1}
\newcommand{\InformationTok}[1]{\textcolor[rgb]{0.56,0.35,0.01}{\textbf{\textit{#1}}}}
\newcommand{\KeywordTok}[1]{\textcolor[rgb]{0.13,0.29,0.53}{\textbf{#1}}}
\newcommand{\NormalTok}[1]{#1}
\newcommand{\OperatorTok}[1]{\textcolor[rgb]{0.81,0.36,0.00}{\textbf{#1}}}
\newcommand{\OtherTok}[1]{\textcolor[rgb]{0.56,0.35,0.01}{#1}}
\newcommand{\PreprocessorTok}[1]{\textcolor[rgb]{0.56,0.35,0.01}{\textit{#1}}}
\newcommand{\RegionMarkerTok}[1]{#1}
\newcommand{\SpecialCharTok}[1]{\textcolor[rgb]{0.00,0.00,0.00}{#1}}
\newcommand{\SpecialStringTok}[1]{\textcolor[rgb]{0.31,0.60,0.02}{#1}}
\newcommand{\StringTok}[1]{\textcolor[rgb]{0.31,0.60,0.02}{#1}}
\newcommand{\VariableTok}[1]{\textcolor[rgb]{0.00,0.00,0.00}{#1}}
\newcommand{\VerbatimStringTok}[1]{\textcolor[rgb]{0.31,0.60,0.02}{#1}}
\newcommand{\WarningTok}[1]{\textcolor[rgb]{0.56,0.35,0.01}{\textbf{\textit{#1}}}}
\usepackage{graphicx}
\makeatletter
\def\maxwidth{\ifdim\Gin@nat@width>\linewidth\linewidth\else\Gin@nat@width\fi}
\def\maxheight{\ifdim\Gin@nat@height>\textheight\textheight\else\Gin@nat@height\fi}
\makeatother
% Scale images if necessary, so that they will not overflow the page
% margins by default, and it is still possible to overwrite the defaults
% using explicit options in \includegraphics[width, height, ...]{}
\setkeys{Gin}{width=\maxwidth,height=\maxheight,keepaspectratio}
% Set default figure placement to htbp
\makeatletter
\def\fps@figure{htbp}
\makeatother
\setlength{\emergencystretch}{3em} % prevent overfull lines
\providecommand{\tightlist}{%
  \setlength{\itemsep}{0pt}\setlength{\parskip}{0pt}}
\setcounter{secnumdepth}{-\maxdimen} % remove section numbering
\usepackage{booktabs}
\usepackage{longtable}
\usepackage{array}
\usepackage{multirow}
\usepackage{wrapfig}
\usepackage{float}
\usepackage{colortbl}
\usepackage{pdflscape}
\usepackage{tabu}
\usepackage{threeparttable}
\usepackage{threeparttablex}
\usepackage[normalem]{ulem}
\usepackage{makecell}
\usepackage{xcolor}
\ifluatex
  \usepackage{selnolig}  % disable illegal ligatures
\fi

\author{}
\date{\vspace{-2.5em}}

\begin{document}

\ifdefined\ifprincipal
\else
\setlength{\parindent}{1em}
\pagestyle{fancy}
\setcounter{tocdepth}{4}
\tableofcontents

\fi

\ifdefined\ifdoblecara
\fancyhead{}{}
\fancyhead[LE,RO]{\scriptsize\rightmark}
\fancyfoot[LO,RE]{\scriptsize\slshape \leftmark}
\fancyfoot[C]{}
\fancyfoot[LE,RO]{\footnotesize\thepage}
\else
\fancyhead{}{}
\fancyhead[RO]{\scriptsize\rightmark}
\fancyfoot[LO]{\scriptsize\slshape \leftmark}
\fancyfoot[C]{}
\fancyfoot[RO]{\footnotesize\thepage}
\fi
\renewcommand{\headrulewidth}{0.4pt}
\renewcommand{\footrulewidth}{0.4pt}

\hypertarget{modelo-de-regresiuxf3n-de-poisson}{%
\subparagraph{Modelo de Regresión de
Poisson}\label{modelo-de-regresiuxf3n-de-poisson}}

Dado que la variable respuesta es discreta y de tipo conteo, se ha
elegido este modelo en el que se asume que el volumen de ventas diario
sigue una distribución de Poisson.

\[Y \sim Po(\mu), \quad \mu=\text{Número medio de ventas diario}\]

Modelado

En primer lugar, estimaremos los parámetros del modelo de regresión de
Poisson, utilizando un conjunto de datos con variables dummy para el día
de la semana y el mes del año, con la intención de representar la
pertenencia de cada instancia a los distintos grupos.

También entrenaremos un modelo haciendo uso del conjunto de datos con
las variables día de la semana y mes en forma de factor, para comprobar
que modelo nos da unas mejores métricas.

Las variables explicativas son las siguientes:

\begin{itemize}
\tightlist
\item
  Variables dummy/factorizadas del día de la semana y el mes del año
\item
  Precio medio con impuestos y descuento
\item
  Día de la semana
\item
  Mes del año
\end{itemize}

\textbf{Ventas totales}

\begin{Shaded}
\begin{Highlighting}[]
\NormalTok{ModeloP\_TOT\_dummy }\OtherTok{=} 
  \FunctionTok{glm}\NormalTok{(VENTAS}\SpecialCharTok{\textasciitilde{}}\NormalTok{PRECIO\_MEDIO\_IMPUESTOS}\SpecialCharTok{*}\NormalTok{DESCUENTO\_MEDIO}\SpecialCharTok{+}
\NormalTok{              (LUNES}\SpecialCharTok{+}\NormalTok{MARTES}\SpecialCharTok{+}\NormalTok{MIERCOLES}\SpecialCharTok{+}\NormalTok{JUEVES}\SpecialCharTok{+}\NormalTok{VIERNES}\SpecialCharTok{+}\NormalTok{SABADO}\SpecialCharTok{+}\NormalTok{DOMINGO)}\SpecialCharTok{+}
\NormalTok{              (AGOSTO}\SpecialCharTok{+}\NormalTok{SEPTIEMBRE}\SpecialCharTok{+}\NormalTok{OCTUBRE}\SpecialCharTok{+}\NormalTok{NOVIEMBRE}\SpecialCharTok{+}\NormalTok{DICIEMBRE}\SpecialCharTok{+}\NormalTok{ENERO),}
      \AttributeTok{family =} \FunctionTok{poisson}\NormalTok{(}\AttributeTok{link =} \StringTok{"log"}\NormalTok{),}
      \AttributeTok{data=}\NormalTok{Ventas\_TOTAL\_ENT\_poiss)}
\FunctionTok{summary}\NormalTok{(ModeloP\_TOT\_dummy)}
\end{Highlighting}
\end{Shaded}

\begin{verbatim}
## 
## Call:
## glm(formula = VENTAS ~ PRECIO_MEDIO_IMPUESTOS * DESCUENTO_MEDIO + 
##     (LUNES + MARTES + MIERCOLES + JUEVES + VIERNES + SABADO + 
##         DOMINGO) + (AGOSTO + SEPTIEMBRE + OCTUBRE + NOVIEMBRE + 
##     DICIEMBRE + ENERO), family = poisson(link = "log"), data = Ventas_TOTAL_ENT_poiss)
## 
## Deviance Residuals: 
##     Min       1Q   Median       3Q      Max  
## -51.810   -6.571    1.080    4.994   64.707  
## 
## Coefficients: (2 not defined because of singularities)
##                                          Estimate Std. Error z value Pr(>|z|)
## (Intercept)                            -32.219952   0.955000 -33.738  < 2e-16
## PRECIO_MEDIO_IMPUESTOS                  26.098720   0.663672  39.325  < 2e-16
## DESCUENTO_MEDIO                          2.655409   3.725073   0.713    0.476
## LUNES1                                   2.503693   0.015988 156.601  < 2e-16
## MARTES1                                  2.415363   0.016009 150.873  < 2e-16
## MIERCOLES1                               2.365094   0.016268 145.386  < 2e-16
## JUEVES1                                  2.595014   0.016010 162.082  < 2e-16
## VIERNES1                                 2.402489   0.016007 150.085  < 2e-16
## SABADO1                                  2.500752   0.016010 156.204  < 2e-16
## DOMINGO1                                       NA         NA      NA       NA
## AGOSTO1                                 -0.254510   0.006886 -36.962  < 2e-16
## SEPTIEMBRE1                             -0.125003   0.006765 -18.479  < 2e-16
## OCTUBRE1                                 0.108339   0.006135  17.660  < 2e-16
## NOVIEMBRE1                              -0.077361   0.007198 -10.748  < 2e-16
## DICIEMBRE1                              -0.034890   0.006418  -5.436 5.44e-08
## ENERO1                                         NA         NA      NA       NA
## PRECIO_MEDIO_IMPUESTOS:DESCUENTO_MEDIO  -1.825982   2.588410  -0.705    0.481
##                                           
## (Intercept)                            ***
## PRECIO_MEDIO_IMPUESTOS                 ***
## DESCUENTO_MEDIO                           
## LUNES1                                 ***
## MARTES1                                ***
## MIERCOLES1                             ***
## JUEVES1                                ***
## VIERNES1                               ***
## SABADO1                                ***
## DOMINGO1                                  
## AGOSTO1                                ***
## SEPTIEMBRE1                            ***
## OCTUBRE1                               ***
## NOVIEMBRE1                             ***
## DICIEMBRE1                             ***
## ENERO1                                    
## PRECIO_MEDIO_IMPUESTOS:DESCUENTO_MEDIO    
## ---
## Signif. codes:  0 '***' 0.001 '**' 0.01 '*' 0.05 '.' 0.1 ' ' 1
## 
## (Dispersion parameter for poisson family taken to be 1)
## 
##     Null deviance: 102206  on 144  degrees of freedom
## Residual deviance:  25019  on 130  degrees of freedom
## AIC: 26371
## 
## Number of Fisher Scoring iterations: 4
\end{verbatim}

\end{document}
