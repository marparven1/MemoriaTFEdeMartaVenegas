\documentclass[12pt,a4paper,]{book}
\def\ifdoblecara{} %% set to true
\def\ifprincipal{} %% set to true
\let\ifprincipal\undefined %% set to false
\def\ifcitapandoc{} %% set to true
\let\ifcitapandoc\undefined %% set to false
\usepackage{lmodern}
\usepackage{amssymb,amsmath}
\usepackage{ifxetex,ifluatex}
%\usepackage{fixltx2e} % provides \textsubscript %PLLC
\ifnum 0\ifxetex 1\fi\ifluatex 1\fi=0 % if pdftex
  \usepackage[T1]{fontenc}
  \usepackage[utf8]{inputenc}
\else % if luatex or xelatex
  \ifxetex
    \usepackage{mathspec}
  \else
    \usepackage{fontspec}
  \fi
  \defaultfontfeatures{Ligatures=TeX,Scale=MatchLowercase}
\fi
% use upquote if available, for straight quotes in verbatim environments
\IfFileExists{upquote.sty}{\usepackage{upquote}}{}
% use microtype if available
\IfFileExists{microtype.sty}{%
\usepackage{microtype}
\UseMicrotypeSet[protrusion]{basicmath} % disable protrusion for tt fonts
}{}
\usepackage[margin = 2.5cm]{geometry}
\usepackage{hyperref}
\hypersetup{unicode=true,
            pdfauthor={Nombre Completo Autor},
              pdfborder={0 0 0},
              breaklinks=true}
\urlstyle{same}  % don't use monospace font for urls
\usepackage{natbib}
\bibliographystyle{plainnat}
\usepackage[usenames,dvipsnames]{xcolor}  %new PLLC
\IfFileExists{parskip.sty}{%
\usepackage{parskip}
}{% else
\setlength{\parindent}{0pt}
\setlength{\parskip}{6pt plus 2pt minus 1pt}
}
\setlength{\emergencystretch}{3em}  % prevent overfull lines
\providecommand{\tightlist}{%
  \setlength{\itemsep}{0pt}\setlength{\parskip}{0pt}}
\setcounter{secnumdepth}{5}
% Redefines (sub)paragraphs to behave more like sections
\ifx\paragraph\undefined\else
\let\oldparagraph\paragraph
\renewcommand{\paragraph}[1]{\oldparagraph{#1}\mbox{}}
\fi
\ifx\subparagraph\undefined\else
\let\oldsubparagraph\subparagraph
\renewcommand{\subparagraph}[1]{\oldsubparagraph{#1}\mbox{}}
\fi

%%% Use protect on footnotes to avoid problems with footnotes in titles
\let\rmarkdownfootnote\footnote%
\def\footnote{\protect\rmarkdownfootnote}


  \title{}
    \author{Nombre Completo Autor}
      \date{27/10/2017}


%%%%%%% inicio: latex_preambulo.tex PLLC

%% UTILIZA CODIFICACIÓN UTF-8
%% MODIFICARLO CONVENIENTEMENTE PARA USARLO CON OTRAS CODIFICACIONES


%\usepackage[spanish,es-nodecimaldot,es-noshorthands]{babel}
\usepackage[spanish,es-nodecimaldot,es-noshorthands,es-tabla]{babel}
% Ver: es-tabla (en: https://osl.ugr.es/CTAN/macros/latex/contrib/babel-contrib/spanish/spanish.pdf)
% es-tabla (en: https://tex.stackexchange.com/questions/80443/change-the-word-table-in-table-captions)
\usepackage{float}
\usepackage{placeins}
\usepackage{fancyhdr}
% Solucion: ! LaTeX Error: Command \counterwithout already defined.
% https://tex.stackexchange.com/questions/425600/latex-error-command-counterwithout-already-defined
\let\counterwithout\relax
\let\counterwithin\relax
\usepackage{chngcntr}
%\usepackage{microtype}  %antes en template PLLC
\usepackage[utf8]{inputenc}
\usepackage[T1]{fontenc} % Usa codificación 8-bit que tiene 256 glyphs

%\usepackage[dvipsnames]{xcolor}
%\usepackage[usenames,dvipsnames]{xcolor}  %new
\usepackage{pdfpages}
%\usepackage{natbib}




% Para portada: latex_paginatitulo_mod_ST02.tex (inicio)
\usepackage{tikz}
\usepackage{epigraph}
\input{portadas/latex_paginatitulo_mod_ST02_add.sty}
% Para portada: latex_paginatitulo_mod_ST02.tex (fin)

% Para portada: latex_paginatitulo_mod_OV01.tex (inicio)
\usepackage{cpimod}
% Para portada: latex_paginatitulo_mod_OV01.tex (fin)

% Para portada: latex_paginatitulo_mod_OV03.tex (inicio)
\usepackage{KTHEEtitlepage}
% Para portada: latex_paginatitulo_mod_OV03.tex (fin)

\renewcommand{\contentsname}{Índice}
\renewcommand{\listfigurename}{Índice de figuras}
\renewcommand{\listtablename}{Índice de tablas}
\newcommand{\bcols}{}
\newcommand{\ecols}{}
\newcommand{\bcol}[1]{\begin{minipage}{#1\linewidth}}
\newcommand{\ecol}{\end{minipage}}
\newcommand{\balertblock}[1]{\begin{alertblock}{#1}}
\newcommand{\ealertblock}{\end{alertblock}}
\newcommand{\bitemize}{\begin{itemize}}
\newcommand{\eitemize}{\end{itemize}}
\newcommand{\benumerate}{\begin{enumerate}}
\newcommand{\eenumerate}{\end{enumerate}}
\newcommand{\saltopagina}{\newpage}
\newcommand{\bcenter}{\begin{center}}
\newcommand{\ecenter}{\end{center}}
\newcommand{\beproof}{\begin{proof}} %new
\newcommand{\eeproof}{\end{proof}} %new
%De: https://texblog.org/2007/11/07/headerfooter-in-latex-with-fancyhdr/
% \fancyhead
% E: Even page
% O: Odd page
% L: Left field
% C: Center field
% R: Right field
% H: Header
% F: Footer
%\fancyhead[CO,CE]{Resultados}

%OPCION 1
% \fancyhead[LE,RO]{\slshape \rightmark}
% \fancyhead[LO,RE]{\slshape \leftmark}
% \fancyfoot[C]{\thepage}
% \renewcommand{\headrulewidth}{0.4pt}
% \renewcommand{\footrulewidth}{0pt}

%OPCION 2
% \fancyhead[LE,RO]{\slshape \rightmark}
% \fancyfoot[LO,RE]{\slshape \leftmark}
% \fancyfoot[LE,RO]{\thepage}
% \renewcommand{\headrulewidth}{0.4pt}
% \renewcommand{\footrulewidth}{0.4pt}
%%%%%%%%%%
\usepackage{calc,amsfonts}
% Elimina la cabecera de páginas impares vacías al finalizar los capítulos
\usepackage{emptypage}
\makeatletter

%\definecolor{ocre}{RGB}{25,25,243} % Define el color azul (naranja) usado para resaltar algunas salidas
\definecolor{ocre}{RGB}{0,0,0} % Define el color a negro (aparece en los teoremas

%\usepackage{calc} 

\usepackage{lipsum}

%\usepackage{tikz} % Requerido para dibujar formas personalizadas

%\usepackage{amsmath,amsthm,amssymb,amsfonts}
\usepackage{amsthm}


% Boxed/framed environments
\newtheoremstyle{ocrenumbox}% % Theorem style name
{0pt}% Space above
{0pt}% Space below
{\normalfont}% % Body font
{}% Indent amount
{\small\bf\sffamily\color{ocre}}% % Theorem head font
{\;}% Punctuation after theorem head
{0.25em}% Space after theorem head
{\small\sffamily\color{ocre}\thmname{#1}\nobreakspace\thmnumber{\@ifnotempty{#1}{}\@upn{#2}}% Theorem text (e.g. Theorem 2.1)
\thmnote{\nobreakspace\the\thm@notefont\sffamily\bfseries\color{black}---\nobreakspace#3.}} % Optional theorem note
\renewcommand{\qedsymbol}{$\blacksquare$}% Optional qed square

\newtheoremstyle{blacknumex}% Theorem style name
{5pt}% Space above
{5pt}% Space below
{\normalfont}% Body font
{} % Indent amount
{\small\bf\sffamily}% Theorem head font
{\;}% Punctuation after theorem head
{0.25em}% Space after theorem head
{\small\sffamily{\tiny\ensuremath{\blacksquare}}\nobreakspace\thmname{#1}\nobreakspace\thmnumber{\@ifnotempty{#1}{}\@upn{#2}}% Theorem text (e.g. Theorem 2.1)
\thmnote{\nobreakspace\the\thm@notefont\sffamily\bfseries---\nobreakspace#3.}}% Optional theorem note

\newtheoremstyle{blacknumbox} % Theorem style name
{0pt}% Space above
{0pt}% Space below
{\normalfont}% Body font
{}% Indent amount
{\small\bf\sffamily}% Theorem head font
{\;}% Punctuation after theorem head
{0.25em}% Space after theorem head
{\small\sffamily\thmname{#1}\nobreakspace\thmnumber{\@ifnotempty{#1}{}\@upn{#2}}% Theorem text (e.g. Theorem 2.1)
\thmnote{\nobreakspace\the\thm@notefont\sffamily\bfseries---\nobreakspace#3.}}% Optional theorem note

% Non-boxed/non-framed environments
\newtheoremstyle{ocrenum}% % Theorem style name
{5pt}% Space above
{5pt}% Space below
{\normalfont}% % Body font
{}% Indent amount
{\small\bf\sffamily\color{ocre}}% % Theorem head font
{\;}% Punctuation after theorem head
{0.25em}% Space after theorem head
{\small\sffamily\color{ocre}\thmname{#1}\nobreakspace\thmnumber{\@ifnotempty{#1}{}\@upn{#2}}% Theorem text (e.g. Theorem 2.1)
\thmnote{\nobreakspace\the\thm@notefont\sffamily\bfseries\color{black}---\nobreakspace#3.}} % Optional theorem note
\renewcommand{\qedsymbol}{$\blacksquare$}% Optional qed square
\makeatother



% Define el estilo texto theorem para cada tipo definido anteriormente
\newcounter{dummy} 
\numberwithin{dummy}{section}
\theoremstyle{ocrenumbox}
\newtheorem{theoremeT}[dummy]{Teorema}  % (Pedro: Theorem)
\newtheorem{problem}{Problema}[chapter]  % (Pedro: Problem)
\newtheorem{exerciseT}{Ejercicio}[chapter] % (Pedro: Exercise)
\theoremstyle{blacknumex}
\newtheorem{exampleT}{Ejemplo}[chapter] % (Pedro: Example)
\theoremstyle{blacknumbox}
\newtheorem{vocabulary}{Vocabulario}[chapter]  % (Pedro: Vocabulary)
\newtheorem{definitionT}{Definición}[section]  % (Pedro: Definition)
\newtheorem{corollaryT}[dummy]{Corolario}  % (Pedro: Corollary)
\theoremstyle{ocrenum}
\newtheorem{proposition}[dummy]{Proposición} % (Pedro: Proposition)


\usepackage[framemethod=default]{mdframed}



\newcommand{\intoo}[2]{\mathopen{]}#1\,;#2\mathclose{[}}
\newcommand{\ud}{\mathop{\mathrm{{}d}}\mathopen{}}
\newcommand{\intff}[2]{\mathopen{[}#1\,;#2\mathclose{]}}
\newtheorem{notation}{Notation}[chapter]


\mdfdefinestyle{exampledefault}{%
rightline=true,innerleftmargin=10,innerrightmargin=10,
frametitlerule=true,frametitlerulecolor=green,
frametitlebackgroundcolor=yellow,
frametitlerulewidth=2pt}


% Theorem box
\newmdenv[skipabove=7pt,
skipbelow=7pt,
backgroundcolor=black!5,
linecolor=ocre,
innerleftmargin=5pt,
innerrightmargin=5pt,
innertopmargin=10pt,%5pt
leftmargin=0cm,
rightmargin=0cm,
innerbottommargin=5pt]{tBox}

% Exercise box	  
\newmdenv[skipabove=7pt,
skipbelow=7pt,
rightline=false,
leftline=true,
topline=false,
bottomline=false,
backgroundcolor=ocre!10,
linecolor=ocre,
innerleftmargin=5pt,
innerrightmargin=5pt,
innertopmargin=10pt,%5pt
innerbottommargin=5pt,
leftmargin=0cm,
rightmargin=0cm,
linewidth=4pt]{eBox}	

% Definition box
\newmdenv[skipabove=7pt,
skipbelow=7pt,
rightline=false,
leftline=true,
topline=false,
bottomline=false,
linecolor=ocre,
innerleftmargin=5pt,
innerrightmargin=5pt,
innertopmargin=10pt,%0pt
leftmargin=0cm,
rightmargin=0cm,
linewidth=4pt,
innerbottommargin=0pt]{dBox}	

% Corollary box
\newmdenv[skipabove=7pt,
skipbelow=7pt,
rightline=false,
leftline=true,
topline=false,
bottomline=false,
linecolor=gray,
backgroundcolor=black!5,
innerleftmargin=5pt,
innerrightmargin=5pt,
innertopmargin=10pt,%5pt
leftmargin=0cm,
rightmargin=0cm,
linewidth=4pt,
innerbottommargin=5pt]{cBox}

% Crea un entorno para cada tipo de theorem y le asigna un estilo 
% con ayuda de las cajas coloreadas anteriores
\newenvironment{theorem}{\begin{tBox}\begin{theoremeT}}{\end{theoremeT}\end{tBox}}
\newenvironment{exercise}{\begin{eBox}\begin{exerciseT}}{\hfill{\color{ocre}\tiny\ensuremath{\blacksquare}}\end{exerciseT}\end{eBox}}				  
\newenvironment{definition}{\begin{dBox}\begin{definitionT}}{\end{definitionT}\end{dBox}}	
\newenvironment{example}{\begin{exampleT}}{\hfill{\tiny\ensuremath{\blacksquare}}\end{exampleT}}		
\newenvironment{corollary}{\begin{cBox}\begin{corollaryT}}{\end{corollaryT}\end{cBox}}	

%	ENVIRONMENT remark
\newenvironment{remark}{\par\vspace{10pt}\small 
% Espacio blanco vertical sobre la nota y tamaño de fuente menor
\begin{list}{}{
\leftmargin=35pt % Indentación sobre la izquierda
\rightmargin=25pt}\item\ignorespaces % Indentación sobre la derecha
\makebox[-2.5pt]{\begin{tikzpicture}[overlay]
\node[draw=ocre!60,line width=1pt,circle,fill=ocre!25,font=\sffamily\bfseries,inner sep=2pt,outer sep=0pt] at (-15pt,0pt){\textcolor{ocre}{N}}; \end{tikzpicture}} % R naranja en un círculo (Pedro)
\advance\baselineskip -1pt}{\end{list}\vskip5pt} 
% Espaciado de línea más estrecho y espacio en blanco después del comentario


\newenvironment{solutionExe}{\par\vspace{10pt}\small 
\begin{list}{}{
\leftmargin=35pt 
\rightmargin=25pt}\item\ignorespaces 
\makebox[-2.5pt]{\begin{tikzpicture}[overlay]
\node[draw=ocre!60,line width=1pt,circle,fill=ocre!25,font=\sffamily\bfseries,inner sep=2pt,outer sep=0pt] at (-15pt,0pt){\textcolor{ocre}{S}}; \end{tikzpicture}} 
\advance\baselineskip -1pt}{\end{list}\vskip5pt} 

\newenvironment{solutionExa}{\par\vspace{10pt}\small 
\begin{list}{}{
\leftmargin=35pt 
\rightmargin=25pt}\item\ignorespaces 
\makebox[-2.5pt]{\begin{tikzpicture}[overlay]
\node[draw=ocre!60,line width=1pt,circle,fill=ocre!55,font=\sffamily\bfseries,inner sep=2pt,outer sep=0pt] at (-15pt,0pt){\textcolor{ocre}{S}}; \end{tikzpicture}} 
\advance\baselineskip -1pt}{\end{list}\vskip5pt} 

\usepackage{tcolorbox}

\usetikzlibrary{trees}

\theoremstyle{ocrenum}
\newtheorem{solutionT}[dummy]{Solución}  % (Pedro: Corollary)
\newenvironment{solution}{\begin{cBox}\begin{solutionT}}{\end{solutionT}\end{cBox}}	


\newcommand{\tcolorboxsolucion}[2]{%
\begin{tcolorbox}[colback=green!5!white,colframe=green!75!black,title=#1] 
 #2
 %\tcblower  % pone una línea discontinua
\end{tcolorbox}
}% final definición comando

\newtcbox{\mybox}[1][green]{on line,
arc=0pt,outer arc=0pt,colback=#1!10!white,colframe=#1!50!black, boxsep=0pt,left=1pt,right=1pt,top=2pt,bottom=2pt, boxrule=0pt,bottomrule=1pt,toprule=1pt}



\mdfdefinestyle{exampledefault}{%
rightline=true,innerleftmargin=10,innerrightmargin=10,
frametitlerule=true,frametitlerulecolor=green,
frametitlebackgroundcolor=yellow,
frametitlerulewidth=2pt}





\newcommand{\betheorem}{\begin{theorem}}
\newcommand{\eetheorem}{\end{theorem}}
\newcommand{\bedefinition}{\begin{definition}}
\newcommand{\eedefinition}{\end{definition}}

\newcommand{\beremark}{\begin{remark}}
\newcommand{\eeremark}{\end{remark}}
\newcommand{\beexercise}{\begin{exercise}}
\newcommand{\eeexercise}{\end{exercise}}
\newcommand{\beexample}{\begin{example}}
\newcommand{\eeexample}{\end{example}}
\newcommand{\becorollary}{\begin{corollary}}
\newcommand{\eecorollary}{\end{corollary}}


\newcommand{\besolutionExe}{\begin{solutionExe}}
\newcommand{\eesolutionExe}{\end{solutionExe}}
\newcommand{\besolutionExa}{\begin{solutionExa}}
\newcommand{\eesolutionExa}{\end{solutionExa}}


%%%%%%%%


% Caja Salida Markdown
\newmdenv[skipabove=7pt,
skipbelow=7pt,
rightline=false,
leftline=true,
topline=false,
bottomline=false,
backgroundcolor=GreenYellow!10,
linecolor=GreenYellow!80,
innerleftmargin=5pt,
innerrightmargin=5pt,
innertopmargin=10pt,%5pt
innerbottommargin=5pt,
leftmargin=0cm,
rightmargin=0cm,
linewidth=4pt]{mBox}	

%% RMarkdown
\newenvironment{markdownsal}{\begin{mBox}}{\end{mBox}}	

\newcommand{\bmarkdownsal}{\begin{markdownsal}}
\newcommand{\emarkdownsal}{\end{markdownsal}}


\usepackage{array}
\usepackage{multirow}
\usepackage{wrapfig}
\usepackage{colortbl}
\usepackage{pdflscape}
\usepackage{tabu}
\usepackage{threeparttable}
\usepackage{subfig} %new
%\usepackage{booktabs,dcolumn,rotating,thumbpdf,longtable}
\usepackage{dcolumn,rotating}  %new
\usepackage[graphicx]{realboxes} %new de: https://stackoverflow.com/questions/51633434/prevent-pagebreak-in-kableextra-landscape-table

%define el interlineado vertical
%\renewcommand{\baselinestretch}{1.5}

%define etiqueta para las Tablas o Cuadros
%\renewcommand\spanishtablename{Tabla}

%%\bibliographystyle{plain} %new no necesario


%%%%%%%%%%%% PARA USO CON biblatex
% \DefineBibliographyStrings{english}{%
%   backrefpage = {ver pag.\adddot},%
%   backrefpages = {ver pags.\adddot}%
% }

% \DefineBibliographyStrings{spanish}{%
%   backrefpage = {ver pag.\adddot},%
%   backrefpages = {ver pags.\adddot}%
% }
% 
% \DeclareFieldFormat{pagerefformat}{\mkbibparens{{\color{red}\mkbibemph{#1}}}}
% \renewbibmacro*{pageref}{%
%   \iflistundef{pageref}
%     {}
%     {\printtext[pagerefformat]{%
%        \ifnumgreater{\value{pageref}}{1}
%          {\bibstring{backrefpages}\ppspace}
%          {\bibstring{backrefpage}\ppspace}%
%        \printlist[pageref][-\value{listtotal}]{pageref}}}}
% 
%%% de kableExtra
\usepackage{booktabs}
\usepackage{longtable}
%\usepackage{array}
%\usepackage{multirow}
%\usepackage{wrapfig}
%\usepackage{float}
%\usepackage{colortbl}
%\usepackage{pdflscape}
%\usepackage{tabu}
%\usepackage{threeparttable}
\usepackage{threeparttablex}
\usepackage[normalem]{ulem}
\usepackage{makecell}
%\usepackage{xcolor}

%%%%%%% fin: latex_preambulo.tex PLLC


\begin{document}


\raggedbottom

\ifdefined\ifprincipal
\else
\setlength{\parindent}{1em}
\pagestyle{fancy}
\setcounter{tocdepth}{4}
\tableofcontents

\fi

\ifdefined\ifdoblecara
\fancyhead{}{}
\fancyhead[LE,RO]{\scriptsize\rightmark}
\fancyfoot[LO,RE]{\scriptsize\slshape \leftmark}
\fancyfoot[C]{}
\fancyfoot[LE,RO]{\footnotesize\thepage}
\else
\fancyhead{}{}
\fancyhead[RO]{\scriptsize\rightmark}
\fancyfoot[LO]{\scriptsize\slshape \leftmark}
\fancyfoot[C]{}
\fancyfoot[RO]{\footnotesize\thepage}
\fi
\renewcommand{\headrulewidth}{0.4pt}
\renewcommand{\footrulewidth}{0.4pt}

\hypertarget{modelos-estaduxedsticos-cluxe1sicos}{%
\chapter{Modelos estadísticos
clásicos}\label{modelos-estaduxedsticos-cluxe1sicos}}

A continuación se exponen los modelos estadísticos clásicos de cara a
predecir la demanda de los productos. Esta varaible depende de varios
factores, como el período del año, el precio del producto, el precio de
los productos competidores o los gustos de cada consumidor, entre otros.
Se trata de una variable cuantitativa discreta, ya que el número de
productos que se venden será un valor entero no negativo
\(y=0,1,\dots\).

\hypertarget{modelo-de-regresiuxf3n-lineal-general}{%
\section{Modelo de Regresión Lineal
General}\label{modelo-de-regresiuxf3n-lineal-general}}

El objetivo es encontrar un modelo estadístico que describa la situación
real de ventas de productos a través de un Modelo Lineal General (MLG),
donde una variable de interés (variable objetivo) pueda ser descrita por
un conjunto de variables explicativas (variables independientes).

Para ello, debemos estimar los parámetros que caracterizan al modelo, es
decir, medir el efecto de cada variable explicativa sobre la variable
objetivo, y con este fin, es necesario definir una serie de hipótesis
del modelo de regresión lineal general:

\begin{itemize}
\item
  Independencia lineal entre las variables explicativas: Esto significa
  que cada variable explicativa contiene información adicional sobre la
  variable objetivo, ya que si hubiera información repetida, habría
  variables explicativas dependientes linealmente de otras.
\item
  Los MLG suponen que existe una función g, llamada función link, que
  relaciona la media de la variable respuesta, \(\mu\) con el resto de
  variables explicativas de la siguiente forma:
  \[E[Y]= \mu = g^{-1}(\eta) = g^{-1}(X^t\beta)\]
\end{itemize}

Siendo:

\begin{itemize}
\item
  Y la variable objetivo
\item
  E(Y) es el valor esperado de la variable Y
\item
  \(\eta = \beta_0 + \beta_1X_1+ \dots + \beta_pX_p= X^t\beta\) es el
  predictor lineal, se trata de una combinación lineal de parámetros
  desconocidos

  \begin{itemize}
  \tightlist
  \item
    \(X_1,\dots,X_p\) son las variables explicativas
  \item
    \(\beta = (\beta_o,\beta_1,\dots,\beta_p)\) representan el efecto de
    cada variable independiente sobre la variable objetivo
  \item
    g es la función link, monótona y diferenciable
  \end{itemize}
\end{itemize}

\hypertarget{componentes-del-modelo-lineal-generalizado}{%
\subsection{Componentes del Modelo Lineal
Generalizado}\label{componentes-del-modelo-lineal-generalizado}}

En este tipo de modelización estadística podemos diferenciar tres
componentes: la componenete aleatoria, la sistemática y la función link
o enlace. Será la combinación de estas tres componentes la que defina
por completo un Modelo Lineal Generalizado.

\hypertarget{componente-aleatoria}{%
\subsubsection{Componente aleatoria}\label{componente-aleatoria}}

Esta componente es la que identifica la variable respuesta y su
distribución de probabilidad.

Sea \(Y\) la variable aleatoria objetivo o variable respuesta objeto de
estudio y sean las \emph{n} variables aleatorias independientes e
idénticamente distribuidas \(Y_1,\dots,Y_n\) la muestra aleatoria
procedente de Y. Siendo \emph{Y} la componente aleatoria cuya
distribución pertenece a la familia exponencial de distribuciones.

En los MLG se supone que la variable respuesta \(Y\) se distribuye de
tal forma que su función de probabilidad, en el caso de estar
modelizando una variable discreta o función de densidad para el caso
continuo viene dada por la sigiente expresión general, conocida como
forma canónica: \[
f(y;\theta, \phi) = a(y,\phi)\cdot e^{\Bigg(\dfrac{y \theta - k(\theta)}{\phi}\Bigg)}
\]

Donde

\begin{itemize}
\tightlist
\item
  \(\theta\) es el parámetro canónico
\item
  \(k(\theta)\) es la función cumulante
\item
  \(\phi > 0\) se trata del parámetro de dispersión
\item
  \(a(y,\phi)\) es una constante normalizadora
\item
  El soporte no depende de \(\theta\) ni de \(\phi\)
\end{itemize}

Además, la media de \(y\) es función del parámetro canónico \(\theta\),
por tanto, se tiene que:

\[
E(Y) = \mu = \dfrac{\partial}{\partial \theta} k(\theta), \\
Var(Y)= \sigma^2 =\phi  \dfrac{\partial^2}{\partial \theta^2} k(\theta) =\phi \dfrac{\partial}{\partial \theta} \Bigg( \dfrac{\partial}{\partial \theta} k(\theta)  \Bigg) = \phi\dfrac{\partial }{\partial \theta}\mu > 0 .
\] Es decir, \(\mu\) es una función estrictamente creciente de
\(\theta\), por lo que estos dos parámetros mantienen una relación
biyectiva.

A \[
V(\mu) = \frac{\partial \mu}{\partial \theta}
\] se le denomina función varianza, por lo que se tiene que:

\[
V(Y) = \phi Var(\mu)
\]

\hypertarget{componente-sistemuxe1tica}{%
\subsubsection{Componente sistemática}\label{componente-sistemuxe1tica}}

Se trata de la coomponente que especifica las variables predictoras
utilizadas en la función predictora lineal en forma de efectos fijos de
un modelo lineal y recoge la variabilidad de la respuesta \emph{Y}
expresada a través de \emph{p} variables explicatibas \(X_1,\dots,X_p\),
que denotamos por \emph{X}, y de los correspondientes parámetros
desconocidos \(\beta=(\beta_0,\beta_1,\dots,\beta_p)'\).

Esta componente, tambien conocida como predictor lineal, viene
representada por \(\eta\) y es una combinación lineal de las variables
explicativas, que viene dada por:
\(\eta =\beta_0 + \beta_1X_1+ \dots + \beta_pX_p= X^t\beta = X^t\beta\)

\hypertarget{funciuxf3n-link-o-funciuxf3n-enlace}{%
\subsubsection{Función link o función
enlace}\label{funciuxf3n-link-o-funciuxf3n-enlace}}

La \emph{función link}: se trata de una función del valor esperado de la
variable respuesta \(E[Y]\), como una combinación lineal de las
variables predictoras. Sin embargo, en muchos casos reales esta relación
no es adecuada, por lo que es necesario la introducción de una función
con el objetivo de relacionar el valor esperado con las variables
explicativas. Por ello, introducimos la función link o función enlace,
\(g(\cdot)\) que relaciona \(\mu\) con el predictor lineal de la
siguiente forma:

\[
g(\mu)=\beta_0 + \beta_1x_1+ \dots + \beta_px_p
\]

En problemas reales, pueden existir varias funciones link, por lo que se
elegirá aquella que facilite la interpretación del modelo óptimo
obtenido. En particular, para cada elemento de la familia exponencial
existe una función enlace denominada función canónica, que permite
relacionar el parámetro canónico con el predictor lineal. \[
\theta_i = \theta(\mu_i) = \eta_i =  X^t_i\beta \quad g(\mu_i)=\theta(\mu_i)
\]

\hypertarget{estimaciuxf3n-de-paruxe1metros}{%
\subsection{Estimación de
parámetros}\label{estimaciuxf3n-de-paruxe1metros}}

Trás la construcción de los modelos, se estiman los parámetros
desconocidos del predictor lineal,
\(\beta=(\beta_o,\beta_1,\dots,\beta_p)\) por
\(\hat\beta=(\hat\beta_o,\hat\beta_1,\dots,\hat\beta_p)\) y el valor del
parámetro de dispersión \(\phi\) por \(\hat\phi\). Posteriormente, se
valora la precisión de las estimaciones con el objetivo de seleccionar
un modelo óptimo.

Generalmente, la estimación de los parámetros se lleva a cabo por el
método de la \emph{Máxima Verosimilitud} o el método de \emph{Mínimos
Cuadrados Ordinarios}. Una vez desarrollados los modelos, se realizará
una comparación de los mismos con el objetivo de seleccionar el mejor.
En el caso del modelado con fines predictivos, se selecionará el modelo
que explique el mayor porcentaje de variabilidad de la respuesta.

Para ello, emplearemos el \textbf{Criterio de información de Akaike
(AIC)}, medida relatica de un modelo estadístico.

Dado un conjunto de modelos candidatos para los datos, el modelo
preferido es aquel que tiene mínimo valor del AIC, que trata de
proporcionar una compensación entre la bondad de ajuste del modelo y la
complejidad del mismo. Es decir, el criterio penaliza al número de
parámetros.

En el caso general, el AIC viene dado por la siguiente expresión:

\[
AIC = 2k - 2ln(\hat L)
\]

Siendo:

\begin{itemize}
\tightlist
\item
  k el número de parámetros del modelo
\item
  \(\hat L\) es el máximo valor de la función de verosimilitud para el
  modelo estimado
\end{itemize}

Otro criterio en el que nos basaremos es en el criterio de bondad de
ajuste, destacando el cálculo del \textbf{coeficiente de determinación}
\(R^2\), que es una medida del grado de fiabilidad o bondad de ajuste
del modelo ajustado a un conjunto de datos. Se trata de una medida
acotada por definición, siendo sus límites \(0 \leq R^2 \leq 1\). Un
coeficiente de determinación igual a 1 indica un ajuste lineal perfecto,
y por tanto, la variación total de la variable \(Y\) es explicada por el
modelo de regresión. Por el contrario, el valor 0 indica que el modelo
no explica nada de la variación total de la variable Y.

Para la bondad de ajuste, otra medida interesante es el RMSE, raíz del
error cuadrático medio. Representa la raíz cuadrada de la distancia
promedio entre el valor real y el pronosticado e indica el ajuste
absoluto del modelo a los datos, es decir, cómo de cerca están los
puntos observados de los valores predichos del modelo. Valores más bajos
de RMSE indican un mejor ajuste.

En muchos casos la variable respuesta es de tipo conteo, como lo es la
variable que queremos modelizar, demanda de productos. Se denominan
variables de recuento o variables de tipo conteo, a aquellas que
determinan el número de sucesos que ocurren en una misma unidad de
observación en un intervalo espacial o temporal definido. Esta variable
\(Y\), puede tomar infinitos números de valores y su probabilidad va en
descenso a medida que sea mayor el valor de la variable.

Para este caso, los modelos que tienen especial interés y que podemos
formalizar a través de modelización lineal son el modelo de
\emph{Poisson} y el modelo de \emph{Binomial negativa}. Estos modelos
permiten analizar el comportamiento de variables de conteo frente a los
valores del conjunto de variables explicativas.

\hypertarget{modelo-de-regresiuxf3n-poisson}{%
\subsection{Modelo de Regresión
Poisson}\label{modelo-de-regresiuxf3n-poisson}}

Se trata del modelo más simple y es el modelo de referencia para
variables respuesta de tipo conteo. Este modelo asume que la variable
respuesta Y sigue una distribución de Poisson, por lo que en el caso de
la modelización de ventas, se define como el número de ventas que
ocurren en un intervalo de tiempo, cuya ocurrencia es aleatoria. Esta
distribución se caracteriza por que la media y varianza coinciden: \[
E(Y)=Var(Y) =\mu 
\]

Se tiene que la distribución de probabilidad de Poisson, y en nuestro
caso, la probabilidad de observar \(y\) ventas es:
\[P(Y=y)= \dfrac{\mu^y e^{-\mu}}{y!}, \quad y=0,1,\dots;\mu>0\]

Y por tanto, la forma canónica o componente aleatoria para esta
distribución es la siguiente:

\[
f(y;\mu) = e^{-\mu}\cdot \frac{\mu^y}{y!} = \frac{1}{y!}e^{ylog(\mu)-\mu}, \quad y \in \{0,1,\dots\}
\]

Es decir, el modelo de Posión se obtiene tomando como función enlace el
parámetro canónico.

donde:

\begin{itemize}
\tightlist
\item
  \(\theta= log(\mu)\) es el parámetro canónico
\item
  \(k(\theta)= \mu=e^\theta\) es la función cumulante
\item
  \(\phi=1\) el parámetro de dispersión
\item
  \(a(y,\phi)= 1/y!\) la constante normalizadora
\end{itemize}

En este caso se tiene que: \(g(\mu_i)=X^t_i\beta\) y una elección usual
de la función link \emph{g} el parámetro canónico, \(g(x)=log(x)\), lo
que equivale a:

\[\mu_i=exp(\beta_o)\cdot\exp(x_{i1}\beta_1)\dots exp(\beta_px_{ip})= exp(\beta_o+\beta_1x_{i1}+\dots + \beta_px_{ip}) \quad \text{ó} \quad log(\mu_i) = \eta_i = X_i\beta\],

así si \(x_i\) se incrementa en una unidad, entonces \(\mu_i\) se
multiplica por \(exp(\beta_i)\). Por tanto, si \(\beta_i>0\), \(\mu_i\)
crece cuando \(x_i\) aumenta y si \(\beta_i<0\), \(\mu_i\) decrece
cuando \(x_i\) aumenta.

Este modelo se ha desarrollado suponiendo que la media y la varianza de
los datos coinciden (equidispersión). Sin embargo, suele ocurrir lo que
se conoce como sobredispersión, es decir, que la varianza es mayor que
la media. Lo habitual es que esta situación se de debido a la existencia
de heterogeneidad entre las observaciones. Cuando esto ocurra,
recurriremos al modelo binomial negativo.

\hypertarget{modelo-de-regresiuxf3n-de-binomial-negativa}{%
\subsection{Modelo de Regresión de Binomial
Negativa}\label{modelo-de-regresiuxf3n-de-binomial-negativa}}

Este modelo es empleado para variables de tipo conteo cuándo existe
sobredispersión, es decir, la media condicional es menor que la varianza
condicional (no coinciden). Existen diferentes modelos binomiales
negativos en función de la variable que se trate de modelar, pero en
este trabajo nos centraremos en el caso de datos de tipo conteo.

La distribución binomial negativa estudia la probabilidad de observar un
número determinado de fracasos \emph{y} (no se producen ventas) , antes
del r-ésimo éxito (se venden r unidades) en una serie de experimentos
Bernoulli independientes, siendo r un número positivo. Se tiene que esta
distribución pertenece a la familia exponencial si el parámetro de
dispersión \(\phi\) es una constante.

Se dice que la variable aleatoria de conteo (número de ventas) \(Y_i\),
con \(i=1,\dots,n\) sigue una distribución Binomial Negativa de
parámetros \emph{r} y \emph{p}, y se representa como
\(Y_i \sim BN(r,p)\) si su función de probabilidad viene dada por: \[
P[Y_i=y_i] =  \genfrac{(}{)}{}{}{y_i+r-1}{r-1} p^r(1-p)^{y_i}
\] donde

\begin{itemize}
\tightlist
\item
  \(0<p<1\)
\item
  \(r>0\)
\item
  \(y_i = 0,1,2,\dots\)
\end{itemize}

Y en este caso, la forma canónica o componente aleatoria para esta
distribución es la siguiente:

\[
f(y;\mu) = exp \Bigg\{y \cdot ln(1-p) + r ln(p) + ln(\genfrac{(}{)}{}{}{y_i+r-1}{r-1}) \Bigg\}
\]

donde

\begin{itemize}
\tightlist
\item
  \(0<p<1\)
\item
  \(r=0,1,2,\dots\)
\item
  \(y_i = 0,1,2,\dots\)
\item
  \(\theta= log(1-p)\) es el parámetro canónico
\item
  \(k(\theta)= -rln(p)=-r ln(1-e^\theta)\) es la función cumulante
\end{itemize}

En este caso la función link es de tipo logarítmico y viene dada por: \[
g(\mu_i)= \theta(\mu_i) = ln \Big( \dfrac{\alpha \mu_i}{1+ \alpha\mu_i}   \Big) = X_i^t\beta = \eta_i 
\]

\hypertarget{anuxe1lisis-de-series-temporales}{%
\section{Análisis de series
temporales}\label{anuxe1lisis-de-series-temporales}}

Aplicaremos este modelo de predicción para tratar de identificar los
patrones de la demanda anterior a lo largo del tiempo y luego proyectar
(predecir) los patrones en el futuro.

Se define una serie temporal como una sucesión de datos ordenados en el
tiempo que corresponden a una misma variable. Los datos son suelen ser
tomados en intervalos regulares de tiempo.

Nuestro objetivo dentro del análisis de series temporales será
identificar el proceso estocástico que ha sido capaz de generar la serie
de estudio.

(No se si añadirlo) Se dice proceso estocástico a una colección o
familia de variables aleatorias \(\{X_t, \text{ con } t \in T \}\) que
siguen la misma ley de distribución y están relacionadas entre sí,
pudiendo por este motivo, describir la información de estas variables en
términos de medias, variaciones y covarianzas.

A continuación encontramos las cuatro etapas en un análisis descriptivo
de series temporales para elegir un modelo que se adecue a nuestros
datos:

\begin{itemize}
\tightlist
\item
  \textbf{Representación gráfica de la serie}. Para tener así una
  primera aproximación del comportamiento de la serie y la existencia de
  posibles tendencias.
\item
  \textbf{Modelización}: Se trata de encontrar el modelo que mejor se
  ajuste a los datos.
\item
  \textbf{Validación de los modelos}: Es necesario saber si el modelo
  ajustado es adecuado o no, por lo que es muy importante el estudio de
  los residuos.
\item
  \textbf{Predicciones}: Una vez construido y validado un modelo,
  realizaremos estimaciones del futuro con nuevas observaciones.
\end{itemize}

En un enfoque clásico de series temporales, asumiremos que el
comportamiento de la variable con respecto al tiempo se compone de
cuatro componentes:

\begin{enumerate}
\def\labelenumi{\arabic{enumi}.}
\item
  \textbf{Tendencia}: Se trata del movimiento suave y regular de la
  serie a largo plazo. La tendencia existe cuando hay un aumento o
  disminución a largo plazo de los datos. Puede ser lineal (ajuste
  mediante una recta) o no lineal (aproximación mediante una curva, como
  por ejemplo logarítmica o exponencial)
\item
  \textbf{Ciclo}: Componente de tipo oscilante caracterizada por
  movimientos recurrentes en torno a la tendencia de la serie y que se
  repiten cada año pero sin una frecuencia fija.
\item
  \textbf{Componente estacional}: Se trata de movimientos regulares
  dentro de la serie con una periodicidad menor a un año, es decir,
  aquello que ocurre generalmente y con la misma intensidad año tras año
  en los mismos períodos, por ejemplo, en la misma época del año o día
  de la semana. Vamos a denotar por L al número de estaciones.
\item
  \textbf{Componente irregular}: Se trata de las variaciones de la serie
  sin un comportamiento sistemático y que no son explicadas por las
  otras tres componentes
\end{enumerate}

Existen diferentes modelos de combinación de las componentes. Para
describir los modelos necesitamos primero una nomenclatura básica.
Denotando por \(X_t\) al valor de la variable en el instante \emph{t},
se tiene: \[ X_t = f(T_t,E_t,I_t)\] donde:

\begin{itemize}
\tightlist
\item
  \(T_t\): Valor de la tendencia en el instante \emph{t}
\item
  \(E_t\): Valor de la componente estacional en el instante \emph{t}
\item
  \(I_t\): Valor de la componente irregular en el instante \emph{t}
  (ruido).
\end{itemize}

Por tanto, los modelos que puede adoptar la función \emph{f} son los
siguientes:

\begin{itemize}
\tightlist
\item
  \textbf{Modelo multiplicativo}: La composición de la serie se realiza
  mediante el producto de sus componentes.
  \[X_t = T_x \times E_t \times I_t\]
\item
  \textbf{Modelo aditivo}: Las componentes se agregan para formar la
  serie temporal. \[X_t = T_x + E_t + I_t\]
\item
  \textbf{Modelo mixto}: La composición de la serie de la parte
  irregular viene de forma aditiva y la parte regular de forma
  multiplicativa. \[Xt=Tx \times Et + It\]
\end{itemize}

\url{http://www5.uva.es/estadmed/datos/series/series2.htm}

Tras haber detectado el modelo mas adecuado, podremos conocer el
comportamiento de la serie a largo plazo.

El siguiente paso realizar una estimación de la tendencia, \(T_t\),
habiendo eliminado previamente la componente estacional para impedir que
estas oscilaciones perturben la identificación de la tendencia.

Para estimar \(T_t\), debemos hacer una hipótesis sobre su forma:

\begin{itemize}
\tightlist
\item
  \textbf{Tendencia determinista}: Se supone que la tendencia es una
  función determinística del tiempo:
  \[T_t = a + bt \quad a,b\in \mathbb R\]
\end{itemize}

Siendo a y b constantes, que se estimarán mediante un modelo de
regresión lineal.

Sin embargo, el método que aplicaremos será el que exponemos a
continuación:

\begin{itemize}
\tightlist
\item
  \textbf{Tendencia evolutiva (método de medias móviles)}: Este método
  consiste en definir la tendencia como una serie suavizada.
  Sunpondremos que la tendencia de la serie es una función que
  evoluciona lentamente y que podremos aproximar función simple del
  tiempo, suponiendo así una recta.
\end{itemize}

Una vez identificada la tendencia, procedemos a hacer un análisis de la
estacionalidad de la serie, con el objetivo de:

\begin{itemize}
\item
  \textbf{Desestacionalizar la serie}, es decir, eliminar las
  oscilaciones periodicas que se repiten a lo largo de los años,
  haciendo así que los datos de distintas estaciones sean comparables.
  La serie desestacionalizada la conseguimos diferenciando la serie.
\item
  \textbf{Realizar predicciones}, ya que si nuestros datos están
  afectados por una componente estacional, necesitaremos una estimación
  de esta de cara a realizar una predicción
\end{itemize}

Para desestacionalizar la serie, emplearemos los índices de variación
estacional asociados a cada estación, ya que se suponen constantes año a
año. Con esta técnica, se evidencian las diferencias en cada período,
por ejemplo, podemos ver la diferencia del volumen de ventas en función
de la época del año (mes, día de la semana, estación,\ldots) Estos
índices reflejan la cantidad fija o proporción en la que se modifica la
tendencia en cada estación.

Una vez calculados estos índices, se desestacionaliza la serie,
eliminando así el efecto de cada estación.

Por último, procedemos a realizar las predicciones. Para ello,
necesitamos que se cumpla la condición de estacionariedad, es decir, la
media y la varianza permanecen constantes en el tiempo (no tiene raíces
unitarias). En el caso de no imponer esta condición de estacionariedad,
predeciríamos carasterísticas que no serán las mismas en el futuro que
en el pasado.

Se tiene que todo proceso lineal es estacionario, por tanto, obtendremos
trabajaremos con series estacionarias, y de lo contrario, podremos
aplicar los mismos métodos a series no estacionarias realizando las
transformaciones pertinentes para conseguir la estacionariedad.

En nuestro caso, aplicaremos la metodología Box-Jenkis como método
predictivo.

\hypertarget{metodologuxeda-box-jenkis}{%
\subsection{Metodología Box-Jenkis}\label{metodologuxeda-box-jenkis}}

Esta metodología tiene en cuenta la dependencia existente entre los
datos, es decir, cada observación en el instante \(t\) será modelada a
partir de los valores pasados. Los modelos se conocen con el nombre de
ARIMA (modelos integrados autorregresivos de medias móviles), que deriva
de las siguientes componentes: AR (Autorregresivo) , I ( integrado),
MA(Medias móviles)

El siguiente paso es identificar el modelo más adecuado a través del
estudio de la función de autocorrelación (FAC) y la función de
autocorrelación parcial (FAP).

Nota: el método recomienda como mínimo 50 observaciones en la serie
temporal.

Fases de la metodología Box-Jenkis:

\begin{enumerate}
\def\labelenumi{\arabic{enumi}.}
\item
  Identificar el la estructura ARIMA que sigue la serie a través del
  estudio de la función de autocorrelación simple (FAS) y la función de
  autocorrelación parcial (FAP). Determinar el modelo arima consiste en
  identidicar los órdenes p y q de su estructura autoregresiva y de
  medias móviles
\item
  Estiamción de parámetros: Una vez tenemos identificado el modelo,
  estimamos los parámetros AR y MA del modelo por el método de máxima
  verosimilitud, obteniendo el error estándar y los resíduos del modelo

  Nota: Es muy importante comprobar que las estimaciones son
  significativamente no nulas.
\item
  Diagnosis del modelo: Comprobamos que los resíduos sigan un proceso de
  ruido blanco mediante el Test de Ljung-Box.

  Si hemos identificado varios modelos y todos ellos pasan la diagnosis,
  nos quedaremos con uno de ellos según el criterio del menor AIC
\item
  Predicción: una vez identificado y validado el mejor modelo, se
  realizan las predicciones con éste.
\end{enumerate}

\bibliography{bib/library.bib,bib/paquetes.bib}


\addcontentsline{toc}{chapter}{Bibliografía}


\end{document}
